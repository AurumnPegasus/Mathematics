\documentclass{article}
\usepackage[utf8]{inputenc}
\usepackage{mathtools}
\DeclarePairedDelimiter{\floor}{\lfloor}{\rfloor}
\usepackage{chngcntr}
\counterwithin*{equation}{section}
\counterwithin*{equation}{subsection}

\title{Greatest Integer Function}
\author{Shivansh Subramanian}
\date{20\textsuperscript{th} May 2020}

\begin{document}

\maketitle

\section{Introduction}

Any number x can be represented as 
\[
    x = \floor*{x} + \theta
\]


Where \(\floor*{x}\) represents integer part of x whereas \(\theta\) represents the fractional part of x

It is given that
\[
    x \geq \floor*{x}
\]

\section{Theorems}
 
\subsection{Legendre Formula}
If n is a positive integer, and p is a prime, then the exponent of the highest power of p that divides n! is
\[
    \sum_{k=1}^{\infty} = \floor{\frac{n}{p^k}}
\]

Proof

Assume p, 2p, 3p . .. tp all exist before n, that implies they all divide n!
\[
    tp \leq n
\]

Which implies that there are exactly \(\floor{\frac{n}{p} }\) multiplies of p withing n!

Similarly we can do this for \(p^2, p^3 . .. p^k\)

And thus we conclude that
\[
    \sum_{k=1}^{\infty} = \floor{\frac{n}{p^k}}
\]

This can be extended to write

\[
    n! = \prod_{p\leq n} p^{\sum_{k=1}^{\infty} \floor{\frac{n}{p^k}}}
\]


\subsection{}

Let F and f be number theoretic function such that
\[F(n) = \sum_{d\vert n} f(d) \]
Then 

\[
    \sum_{n=1}^{N}F(n) = \sum_{k=1}^{N} f(k) \times \floor{\frac{N}{k}}
\]

Proof:

We know that

\begin{equation}\label{1_2.2}
    \sum_{n=1}^{N} F(n) = \sum_{n=1}^{N} \sum_{d\vert n} f(d)
\end{equation}


Let there be some \(k \leq N\), therefore there must be multiples of k occurring \indent less than N which are \(k, 2k . . . \floor{\frac{N}{k}}\)

So if we can find number of times each f(k) occurs, what that would mean \indent is that we would be finding number of times each number less than N occurs \indent in the second summation of RHS

Thus we can transform equation \ref{1_2.2} into 
\[
    \sum_{n=1}^{N} F(n) = \sum_{n=1}^{N} \sum_{d\vert n} f(d) = \sum_{k=1}^{N} f(k) \times \floor{\frac{N}{k}}
\]

\subsection{}
Let n be represented in p-nary, such that

\[
    n = a_{k}p^{k} + a_{k-1}p^{k-1} + . . . + a_{0}
\]
The exponent of highest power of p occurring in n! is
\[
    \frac{n - (a_k + a_{k-1} + . . + a_0)}{p-1}
\]

Proof:

\[
    answer = \sum_{k=1}^{\infty} \floor{\frac{n}{p}}
\]

\[
    answer = ( a_{k}p^{k-1} + a_{k-1}p^{k-2} + . . a_1 ) + ( a_{k}p^{k-2} + . . . + a_2 ) . . . . ( a_{k} ) 
\]

\[
    answer = a_{k} ( p^{k-1} + p^{k-2} + . . 1 ) + a_{k-1} ( p^{k-2} + p^{k-3} + . . 1 ) . . a_{1}
\]

\[
    answer = a_{k} \frac{p^{k} - 1}{p - 1} + a_{k-1} \frac{p^{k-1} - 1}{p - 1} + . . .  + a_1 \frac{p-1}{p-1} + a_0\frac{1 - 1}{p - 1}
\]

\[
    answer = \frac{n - (a_k + a_{k-1} + .  . . + a_0 )}{p-1}
\]

\end{document}

