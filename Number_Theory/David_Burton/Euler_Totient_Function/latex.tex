\documentclass{article}
\usepackage[utf8]{inputenc}
\usepackage{mathtools}
\usepackage{amssymb}
\DeclarePairedDelimiter{\floor}{\lfloor}{\rfloor}
\usepackage{chngcntr}
\counterwithin*{equation}{section}
\counterwithin*{equation}{subsection}

\title{Euler Totient Function}
\author{ Shivansh Subramanian }
\date{22\textsuperscript{nd} May 2020}

\begin{document}

\maketitle

\section{Introduction}
\( \phi(x) \) is defined as the number of integers d less than or equal to x such that gcd(d, x) = 1

\section{Theorems}
\subsection{}

\[
    \phi(p) = p -1 
\]

\subsection{}
\[
    \phi(p^k) = p^k - p^{k-1} = p^{k} \times [1- \frac{1}{p}]
\]

Proof

We know that for gcd(n, \(p^k\)) = 1, n must not be divisible by p

\(
    1p, 2p, . . . (p^{k-1})p\) are all divisible by p
    
Therefore, there are p\textsuperscript{k-1} numbers divisible by p, hence 

\[
    \phi(p^k) = p^k - p^{k-1} = p^{k} \times [1- \frac{1}{p}]
\]


\subsection{Multiplicativeness}
If gcd(m,n) = 1, then 
\[
    \phi(mn) = \phi(m) \phi(n)
\]

Proof: 

All numbers between 1 and mn can be written as 

$$
\begin{matrix}
    1 & 2 & . &. &. & r &. &. & m           \\
    {m+1} & {m+2} & . & . & . & {m+r} & . & . & {2m}            \\
    . & . & . &. &. & . &. &. & .           \\
    . & . & . &. &. & . &. &. & .           \\
    {(n-1)m+1} & {(n-1)m+2} & . & . & . & {(n-1)m+r} & . & . & {nm}            \\
\end{matrix}
$$

We know \(\phi(mn)\) is equal to number of entries in this array such that the \indent element is relatively prime to mn

\[
    gcd(km + r, m) = gcd(r, m)
\]

This implies that the numbers in a given column are relatively prime to m \indent iff r is relatively prime to m, therefore \(\phi(m)\) such numbers exist in each row

Now in such a row where gcd(r,m) = 1, we have n elements
\[
    r , m + r, 2m + r, . . . (n-1)m + r
\]

Here, we have to show that there exist no two numbers such that they give \indent the same remainder on being divided by n

Assume 
\[
    km + r \equiv (lm + r) mod n
\]

\[
    km \equiv (lm) mod n
\]

Since m and n are relatively prime

\[
    k \equiv l mod n
\]

Therefore it is not possible for different k and l to give same remainder. 

This implies that all the elements in row map to 1 , 2 . . (n-1) as modulo n

Say element s maps to t modulo n
\[
    s \equiv t mod n
\]

We can easily prove that gcd(s, n) = 1 iff gcd(t, n) = 1

Which implies \(\phi(n)\) such integers exist in each row

Hence we conclude that

\[
    \phi(mn) = \phi(m) \phi(n)
\]

\subsection{}

If
\[
    n = p_1^{k_1} p_2^{k_2} . . . p_r^{k_r}
\]
Then
\[
    \phi(n) = n \prod_{x = 1}^{r} [1 - \frac{1}{p_x}]
\]

Proof: 

Since we know \(\phi(n)\) is multiplicative
\[
    \phi(n) = \prod_{x=1}^{r} \phi(p_x^{k_x})
\]

From Theorem 2.2 
\[
    \phi(n) = \prod_{x=1}^r p_x^{k_x} \times [1 - \frac{1}{p_x}]
\]

\[
    \phi(n) = n \prod_{x = 1}^{r} [1 - \frac{1}{p_x}]
\]

\subsection{}
For any positive integer n

\[
    \sqrt{\frac{n}{2}} \leq \phi(n) \leq n
\]

Proof:

The part of proving \(\phi(n) \leq n \) is trivial, we look at the other part then

Let
\[
    n = p_1 p_2 . . . p_k q_1^{a_1} q_2^{a_2} . . q_l^{a_l}
\]

Let 
\[
    s = p_1 p_2 . . p _k
\]
\[
    t = q_1^{a_1} q_2^{a_2} . . q_l^{a_l}
\]

We know
\[
    \phi(n) = \phi(st) = \phi(s)\phi(t)
\]

\[
    \frac{\phi(n)}{\sqrt{n}} = \frac{\phi(s)}{\sqrt{s}} \frac{\phi(t)}{\sqrt{t}}
\]

Then
\[
    \frac{\phi(s)}{\sqrt{s}} = \prod_{x=1}^{k} \frac{p_x - 1}{\sqrt{p_x}}
\]

For all p \(\geq\) 2 , we have \(\frac{p - 1}{\sqrt{p}}\) \(>\) 1

Therefore there are two cases, where m contains only 2 and where n contains \indent 2 and some other factors

The former cases is trivial, and since in the latter case the fraction for other \indent factors other than 2 is greater than 1, we can easily prove
\[
    \frac{\phi(s)}{\sqrt{s}} \geq \frac{1}{\sqrt{2}}
\]
\[
    \frac{\phi(t)}{\sqrt{t}} = \prod_{x=1}^{l} q_x^{a_x - 1} (q_x - 1) \geq 1
\]

Therefore we arrive at the conclusion

\[
    \sqrt{\frac{n}{2}} \leq \phi(n) \leq n
\]

\subsection{}
If 
\[
    n = p_1^{a_1} p_2^{a_2} . . p_r^{a_r}
\]
Then
\[
    \phi(n) \geq \frac{n}{2^r}
\]

Proof:

Since we know \(\phi\) is multiplicative
\[
    \phi(n) = \phi(p_1^{a_1}) \phi(p_2^{a_2}) . . . \phi(p_r^{a_r})
\]

Since for any p
\[
    \phi(p^{a}) = p^a [1 - \frac{1}{p}]
\]

We know that
\[
    \frac{1}{2} \geq \frac{1}{p}
\]

Therefore
\[
    \phi(p^a) = p^a [1 - \frac{1}{p}] \geq \frac{p^a}{2}
\]

\[
    \phi(n) \geq \frac{n}{2^r}
\]


\subsection{}

If n is a composite number, 

\[
    \phi(n) \leq n - \sqrt{n}
\]

Proof:

Let p be the smallest prime divisor of n, then 

Let
\[
    n = p_1^{a_1} p_2^{a_2} . . . p_r^{a_r}
\]

Where p = p\textsubscript{1}
Then
\[
    \phi(n) = n \prod_{k=1}^{r} [1 - \frac{1}{p_k}]
\]

Assume such a n' that p \(\nmid\) n.
Then
\[
    \phi'(n') = n' \prod_{k=2}^{r} [1 - \frac{1}{p_k}]
\]

And
\[
    \phi'(n') \leq n'
\]

\[
    \phi'(n') \times p^{a_1} \leq n' \times p^{a_1}
\]

\[
    \phi'(n') \times p^{a_1} \leq n
\]

\[
    \phi'(n') \times p^{a_1} \times [1 - \frac{1}{p}] \leq n \times [1 - \frac{1}{p}]
\]

\[
    \phi(n) \leq  n \times [1 - \frac{1}{p}]
\]

And since 

\[
    p \leq \sqrt{n}
\]

\[
    \phi(n) \leq  n \times [1 - \frac{1}{p}] \leq n \times [1 - \frac{1}{\sqrt{n}}]
\]

\[
    \phi(n) \leq n - \sqrt{n}
\]

\end{document}

