\documentclass{article}
\usepackage[utf8]{inputenc}
\usepackage{amsmath}

\title{ Chapter 1 }
\author{ S. Shivansh }
\date{ 6\textsuperscript{th} May 2020 }

\usepackage{natbib}
\usepackage{graphicx}

\begin{document}

\maketitle

\counterwithin*{equation}{subsubsection}

\section{ Binomial Theorem }
The Binomial Coefficient is defined by the expression : 
\[
    \binom{n}{k} = \frac{n!}{k!(n-k)!}
\]


\section*{ Pascal's Rule }
\[
    \binom{n}{k} + \binom{n}{k-1} = \binom{n+1}{k}
\]
This relation gives rise to the famous Pascal's Triangle, in which \( \binom{n}{k}\) appears at ( k + 1 )\textsuperscript{th} index of the n\textsuperscript{th} row

\subsection*{ Properties }
\subsubsection{}\label{1.0.1}
This property can be proved by expansion
\[
    \binom{n}{k} \times \binom{k}{r} = \binom{n}{r} \times \binom{n-r}{k-r}
\]
\subsubsection{}\label{1.0.2}
This property can be proved by using Property \ref{1.0.1}
\[
    \binom{n}{k} = \frac{(n-k+1)}{k} \times \binom{n}{k-1}
\]
\subsubsection{}\label{1.0.3}
This property can be proved by expansion. This is valid for $n\geq$4 .
\[
    \binom{n}{k} = \binom{n-2}{k-2} + 2\binom{n-2}{k-1} + \binom{n-2}{k}
\]
\subsubsection{}\label{1.0.4}
\[ \binom{2m}{2} = 2\times\binom{m}{2} + m^2\]

\section{ Triangular Numbers }
A triangular number is a positive integer n such that n = 1 + 2 + .. + k
\subsection*{ Properties }
\subsubsection{}\label{2.0.1}
For x to be triangular, x has to be of the form \(\frac{n\times(n+1)}{2}\)\newline

Proof:\\

\begin{equation}\label{first_equation_2.0.1}
    x = n + (n-1) + (n-2) . . . + 1
\end{equation}
\begin{equation}\label{second_equation_2.0.1}
    x = 1 + 2 + 3 . . . + n
\end{equation}
Adding both \ref{first_equation_2.0.1} and \ref{second_equation_2.0.1}, we get
\begin{equation}
    2 \times x = (n+1) \times n
\end{equation}
From this, we can derive that
\[
    x = \frac{(n+1) \times n}{2}
\]
\subsubsection{}\label{2.0.2}
Integer x is triangular iff 8x + 1 is a perfect square\\
Part 1 : \\

Prove that if x is a triangular number, then 8x + 1 is a perfect square \\

From Property \ref{2.0.1} we know that 
\[ x = \frac{n \times (n+1) }{2}\]
Hence, we know
\[ 2x = n \times (n+1) \]
\[ 8x = 4n^2 + 4n \]
\[ 8x + 1 = 4n^2 + 4n + 1 \]
Hence we get at the final result
\begin{equation}\label{first_equation_2.0.2}
    8x + 1 = ( 2n + 1 )^2
\end{equation}
\newpage
Part 2 : \\

Prove that if 8x + 1 is a perfect square, x is triangular \\
\[ 8x + 1 = k^2 \]

Since LHS is odd, this implies that k is odd. So let k = \( 2n + 1\)
\[ 8x + 1 = (2n + 1)^2 \]
\[ 8x + 1 = 4n^2 + 4n + 1 \]
\[ 8x = 4n^2 + 4n \]
\[ 2x = n^2 + n \]
\begin{equation}\label{second_equation_2.0.2}
    x = \frac{n \times (n+1) }{2}
\end{equation}
From Part 1 and Part 2, we can deduce that the property mentioned is true.

\subsubsection{}\label{2.0.3}
Property of any 2 consecutive triangular numbers is a perfect square. \\


From \ref{2.0.1} we can say
\begin{equation}\label{first_equation_2.0.3}
    x = \frac{n \times (n+1)}{2}
\end{equation}
\begin{equation}\label{second_equation_2.0.3}
    y = \frac{(n+1) \times (n+2) }{2}
\end{equation}
Where x and y are consecutive triangular numbers. \\

From \ref{first_equation_2.0.3} adding \ref{second_equation_2.0.3} \\
\[ x + y = \frac{ (n+1) \times ( n + (n + 2) )}{2}\]
\[ x + y = (n+1)^2 \]

\subsubsection{}\label{2.0.4}
If number x is triangular, then 9x + 1, 25x + 3, 49x + 6 are also triangular.\\

From Property \ref{2.0.2}, we know
\[ x = 8n + 1 \]
Where x is the n\textsuperscript{th} triangular number. 
Now, replacing n with 9n + 1, we get
\[ 8\times(9n+1) + 1\]
\[ 72n + 9 \]
\[ 9(8n + 1)\]
We can observe this is a perfect square by property \ref{2.0.2}. Hence this is a triangular number. We can do similarly for rest of it.
\newpage
\subsubsection{}\label{2.0.5}
if t\textsubscript{n} is the n\textsuperscript{th} triangular number, then
\[ t_1 + t_2 . . . +t_n = \frac{n\times(n+1)\times(n+2)}{6}\]
Proof: \\
Multiplying and dividing LHS by 2, we get
\[\frac{2t_1 + 2t_2 + 2t_3 . . . 2t_n}{2}\]
Rearranging the terms
\[\frac{t_1 + (t_1 + t_2 ) + (t_2 + t_3) . . . (t_{n-1} + t_n ) + t_n}{2} \]
Using Property \ref{2.0.3}
\[\frac{ t_1 + 2^2 + 3^2 . . . n^2 + t_n}{2} \]
\[\frac{1^2 + 2^2 + 3^2 . . . n^2}{2} + \frac{n(n+1)}{4} \]
\begin{equation}\label{first_equation_2.0.5}
  \frac{n(n+1)(2n+1)}{12}  \frac{n(n+1)}{4}  
\end{equation}
Simplifying \ref{first_equation_2.0.5}, we get
\[\frac{n\times(n+1)\times(n+2)}{6}\]
\subsubsection{}\label{2.0.6}
\[1\times2 + 2\times3 + 3\times4 . . . n\times(n+1) = \frac{n(n+1)(n+2)}{3}\]
Proof\\


We use the knowledge that \(2\times \binom{m}{2} = m^2 \)
\[2\times\binom{2}{2} + 2\times\binom{3}{2} . . . 2\times\binom{n+1}{2}\]

Taking 2 common and using Property \ref{2.0.5}, we get this simplified to 
\[\frac{n(n+1)(n+2)}{3}\]

\subsubsection{}\label{2.0.7}
\[\binom{2}{2} + \binom{4}{2} + \binom{6}{2} . . . \binom{2n}{2} = \frac{n(n+1)(4n+1)}{6}\]
Proof\\

By Property \ref{1.0.4}, we get
\[\binom{2\times1}{2} + \binom{2\times2}{2} + \binom{2\times3}{2} . . . \binom{2\times n}{2}\]
\[2\times\binom{1}{2} + 1^2 + 2\times\binom{2}{2} + 2^2 . . . 2\times\binom{n}{2} +n^2\]\

Grouping
\[2( \binom{1}{2} + \binom{2}{2} + \binom{3}{2} . . \binom{n}{2} ) + ( 1^2 + 2^2 + 3^2 . .  + n^2 ) \]
\[2\times\frac{(n-1)n(n+1)}{6} + \frac{n(n+1)(2n-1)}{6}\]

Simplifying this, we get
\[\frac{n(n+1)(4n+1)}{6}\]

\subsubsection{}\label{2.0.8}
\[1^2 + 3^2 + 5^2 . . . (2n-1)^2 = \binom{2n+1}{3}\]
Proof: \\

Using Property \ref{2.0.3}
\[ k^2 = t_{k-1} + t_k\]

For all k belonging to odd integers specified above

\[t_1 + t_2 + t_3 . . . t_{2n-2} + t_{2n-1}\]

Using Property \ref{2.0.5}
\[\frac{2n(2n+1)(2n-1)}{6}\]

Which is nothing but
\[\binom{2n+1}{3}\]

\end{document}

